\documentclass{report}
\usepackage{cours}

\title{Optimization of simplified calculation methods for early age cracking assessment}
\author{Edgar Pierre BURKHART}
\date{2020}

\bibliography{bib}

\begin{document}

\maketitle

\tableofcontents
%%%


\chapter{Literrature review}

\section{Introduction}

Cracking control is a critical point in the design of concrete structures.
Uncontrolled cracking can have a major impact on the durability of reinforced
concrete structures, for several reasons including corrosion of metal rods.
For this reason, the risks of cracking during the early age of a structure have
been studied by numerous research teams in the past. In this chapter, the
results that have been found regarding the evaluation of cracking risks will be
observed.

\section{Estimation of cracking risk}
Several methods have been established in order to estimate the cracking risk of
concrete elements. In this section will be reviewed several methods used in
previous research.

In 2004, a state-of-the-art report on the control of early age cracking was
conducted by a team from the Japan Concrete Institute \cite{soa}. The goal of
this paper was to review existing research on the causes of cracking and
methodologies used to control this phenomenon.

According to this document, cracking is mainly caused by different types of
deformation, citing the main causes of those to be autogenous shrinkage, drying
shrinkage and thermal shrinkage. Although both represent a variation in
relative humidity, they are still separate phenomenons.
Autogenous shrinkage defines the decrease in
material volume due to hydration of cement, while drying shrinkage represents
the volume changed caused by water evaporation in the concrete.
Thermal shrinkage is caused by the variations in temperature in the concrete
caused by the heat generated by the chemical reaction of cement hydration.

It is also pointed that the resulting strain which should be considered is the
result of a combination of elastic strain, the effects of creep, of autogenous
and drying shrinkage, as well as thermal shrinkage, and proposes an equation
defining the total strain as the sum of all strains.

This study also focuses on the characteristics of concrete that matter in
cracking control, and shows the importance of using the correct elastic modulus
and strength in order not to overestimate the cracking risk. It also emerges
from several studies that the specific tensile creep is different from the
specific compressive creep, but their doesn't seem to be a consensus on how to
determine it accurately. It also appears that the linear expansion coefficient
was highly time dependent, and was generally higher at early age.

Multiple analytical models are also presented for the differents aspects of
this type of study, as using the Burgers model to study elastic and creep
strain for instance.

Several methods of cracking behavior testing are also discussed, highlighting
the different experimental approaches possible. A criteria on crack initiation
is also established, allowing for adapted material choices in concrete
structures designing.

Other studies have been conducted in order to investigate the effects of
shrinkage and creep in concrete, as in 2001 by a team of the University of
Illinois \cite{cscea}. Several tests were conducted, using a combination of a
restrained shrinkage test and a free shrinkage test to be able to extract the
creep strain from the experimental study. Their findings show that the history
of stress evolution has a major impact on crack developement. It is also
highlighted that the cracking stress is generally lower than the static tensile
stress, at around \SI{80}{\percent} of the latter. The importance of creep is
also exposed, displaying a creep to shrinkage ratio of over \SI{50}{\percent}.

On the subject of fiber reinforced concrete, the study shows a delay on the
appearance of cracks with the addition of fibers, linking these results to a
likely change in drying creep mechanisms.

%%%
\nocite{*}
\printbibliography
\end{document}
